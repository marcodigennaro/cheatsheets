%\documentclass[10pt]{article}
\documentclass[10pt]{revtex4}

\usepackage{fancyhdr}
%\usepackage{draftcopy}
\usepackage{lscape}
\usepackage{longtable}
\usepackage{amsmath}
%\usepackage[french]{babel}
%\selectlanguage{french}
\usepackage[english]{babel}
\selectlanguage{english}
\usepackage{graphicx}
\include{psfig}
\usepackage{float}
%%%%%%%%%%%%%%%%%%%%%%%%%%%%%%%%%%%%%%%%%%%%%%%%%%%%%%%%%%%%%%
%%Fancy Header%%%%%%%%%%%%%%%%%%%%%%%%%%%%%%%%%%%%%%%%%%%%%%%%
%%%%%%%%%%%%%%%%%%%%%%%%%%%%%%%%%%%%%%%%%%%%%%%%%%%%%%%%%%%%%%
\pagestyle{fancy}
%\renewcommand{\chaptermark}[1]{\markboth{\MakeUppercase{#1}}{}}
\renewcommand{\sectionmark}[1]{\markright{\thesection\ #1}}
\fancyhf{}
\fancyhead[LE,RO]{\bfseries\thepage}
\fancyhead[LO]{\bfseries\rightmark}
\fancyhead[RE]{\bfseries\leftmark}
\renewcommand{\headrulewidth}{0.5pt}
\renewcommand{\footrulewidth}{0pt}
%\addtolength{\headheight}{0.5pt}
\fancypagestyle{plain}
{
    \fancyhead{}
    \renewcommand{\headrulewidth}{0cm}
}

%%%%%%%%%%%%%%%%%%%%%%%%%%%%%%%%%%%%%%%%%%%%%%%%%%%%%%%%%%%%%%
%%Mise en page%%%%%%%%%%%%%%%%%%%%%%%%%%%%%%%%%%%%%%%%%%%%%%%%
%%%%%%%%%%%%%%%%%%%%%%%%%%%%%%%%%%%%%%%%%%%%%%%%%%%%%%%%%%%%%%
\evensidemargin -0.5cm                                        
\topmargin -0.5cm                                                
\headheight 1cm                                               
\headwidth 16cm                                               
\headsep 1cm                                                  
\textheight 22cm                                              
\textwidth 16cm                                               
\marginparsep 0cm                                             
\marginparwidth 0cm                                           
\oddsidemargin 0.5cm                                         




\begin{document}


\title{Testing of JTH-v1.1 PAW table for ABINIT}

%% use optional labels to link authors explicitly to addresses:
%% \author[label1,label2]{<author name>}
%% \address[label1]{<address>}
%% \address[label2]{<address>}

\author{Fran\c{c}ois Jollet}
\author{Marc Torrent }
\affiliation{ CEA, DAM, DIF F-91297 Arpajon, France}
\author{ Natalie Holzwarth}
\address{Department of Physics, Wake Forest University, Winston-Salem, NC 27109 USA}

\date{\today}


\begin{abstract}
%% Text of abstract
A new version of the JTH table (JTHv1.1) is now available. It has been tested both against the $\Delta$ and $\Delta_1$ factors (\cite{Lejaeghere},\cite{Jollet}) and against the lattice parameter of fcc, bcc, rocksalt, perovskite, zinc-blende and half-heusler structures, following the GBRV testing suite \cite{Garrity}.
\end{abstract}

\maketitle

%%
%% Start line numbering here if you want
%%
% \linenumbers

%% main text
\section{JTH-v1.1 PAW table for ABINIT}

A new version of the JTH table (JTHv1.1) is now available. It has been generated with the code ATOMPAW (v4.0.0.12) \cite{web4}. This new version follows the XML format defined in \cite{web2}.\\
Compared with the JTHv1.0 version, the following changes have been made:\\
- the data files have been slightly changed for: H, Li, Si, Cu, Zn, Ga, Cd, Sb, Lu, Os, Ir and Bi.\\
These changed have been made for numerical reasons (generally big ratios between the amplitude of projectors and the amplitude of wavefunctions).
With these modifications, we obtain a table that gives good results for all the tests performed. These results are very similar to the physical results of the JTH-v1.0 table

\section{PAW atomic data validation against the delta factor}

We have used the delta calculation package (version 3.1 \cite{web1}) to validate our new atomic data against the Wien2k code. The electronic structure calculations have been performed thanks to the ABINIT-8.4.4 code \cite{Abinit16}. For this, we have used the recommended values \cite{Lejaeghere} for the k-point sampling (6750/N k-points in the Brillouin zone for a N-atom cell). A Fermi-Dirac broadenning of 0.002 Ha has been used. As indicated in \cite{Lejaeghere}, we have used the cystallographic data (CIF's files) provided with the delta calculation package. The Equation of State (EOS) of each element has been adjusted to a Birch-Murnaghan one thanks to seven calculations at seven different volumes, ranging from 0.94 to 1.06 $V_{\rm S}$, where $V_{\rm S}$ is the equilibrium volume deduced from the CIF's file, without geometry optimisation to be exactly in the same conditions as the Wien2k calculations.\\

The version 3.1 of the delta calculation package gives both the delta factor and the modified delta1 factor proposed by \cite{Jollet}.
For the JTH-v1.1 table, we obtain the mean values of $\Delta$=0.44 meV and $\Delta_1$=1.03 meV for a 20 Ha cutoff, which are very good results (see \cite{web1} for comparison with other PAW data).\\

\newpage
The detailed results for each elements are given here:\\

\begin{verbatim}

#--------------------
# Delta values of Abinit_delta_JTH32-20.txt with respect to WIEN2k.txt (in meV/atom)
# (71 elements of 71 included)
# calculated with calcDelta.py version 3.1 
# from left to right: Delta [meV/atom] - relative Delta [%] - Delta1 [meV/atom]
#--------------------
 H	 0.237	 26.1	3.972 
He	 0.008	 10.8	1.662 
Li	 0.011	  0.8	0.119 
Be	 0.094	  1.9	0.289 
 B	 0.244	  2.8	0.426 
 C	 0.146	  1.2	0.180 
 N	 0.488	  6.1	0.927 
 O	 0.242	  5.0	0.763 
 F	 0.220	  6.5	0.984 
Ne	 0.010	  5.6	0.891 
Na	 0.498	 33.9	5.190 
Mg	 0.256	  6.1	0.934 
Al	 0.098	  1.5	0.229 
Si	 0.365	  4.0	0.604 
 P	 0.488	  6.6	1.003 
 S	 0.304	  4.2	0.636 
Cl	 0.062	  1.7	0.252 
Ar	 0.022	 11.4	1.670 
 K	 0.069	  5.1	0.778 
Ca	 0.111	  3.0	0.459 
Sc	 0.019	  0.3	0.044 
Ti	 1.252	 12.7	1.930 
 V	 1.688	 13.6	2.074 
Cr	 0.723	  6.7	1.024 
Mn	 0.892	 13.6	2.050 
Fe	 0.557	  5.0	0.763 
Co	 1.068	  9.0	1.373 
Ni	 1.457	 13.3	2.028 
Cu	 0.286	  3.3	0.504 
Zn	 0.319	  5.5	0.840 
Ga	 0.126	  2.5	0.379 
Ge	 0.555	  7.7	1.181 
As	 0.494	  6.3	0.959 
Se	 0.238	  3.3	0.510 
Br	 0.109	  2.4	0.369 
Kr	 0.021	  9.3	1.441 
Rb	 0.288	 22.4	3.415 
Sr	 0.755	 23.7	3.616 
 Y	 0.332	  4.8	0.735 
Zr	 0.245	  2.2	0.336 
Nb	 0.193	  1.2	0.188 
Mo	 1.631	  7.8	1.195 
Tc	 1.031	  4.7	0.715 
Ru	 0.344	  1.6	0.241 
Rh	 0.944	  5.1	0.781 
Pd	 1.161	  8.9	1.360 
Ag	 0.215	  2.6	0.397 
Cd	 0.088	  1.7	0.263 
In	 0.209	  4.2	0.641 
Sn	 0.106	  1.6	0.239 
Sb	 0.265	  3.3	0.497 
Te	 0.069	  0.9	0.131 
 I	 0.727	 15.2	2.319 
Xe	 0.010	  4.2	0.645 
Cs	 0.115	  9.9	1.500 
Ba	 0.739	 26.6	4.080 
Lu	 0.343	  4.9	0.750 
Hf	 0.192	  1.6	0.239 
Ta	 0.434	  2.4	0.362 
 W	 1.349	  5.4	0.828 
Re	 0.886	  3.2	0.488 
Os	 0.412	  1.4	0.217 
Ir	 0.622	  2.4	0.368 
Pt	 2.105	 10.6	1.616 
Au	 1.108	  8.6	1.323 
Hg	 0.144	 10.9	1.704 
Tl	 0.056	  1.3	0.201 
Pb	 0.293	  4.5	0.690 
Bi	 0.128	  1.6	0.245 
Po	 0.194	  2.2	0.338 
Rn	 0.023	  8.8	1.355 
#--------------------
#np.mean  0.444	 6.8	1.035
#np.std   0.459	 6.6	1.009
#np.max   2.105	 33.9	5.190 	 (Pt, Na, Na)
#np.min   0.008	 0.3	0.044 	 (He, Sc, Sc)
#--------------------

\end{verbatim}

We have also studied the convergency of the delta factor with the energy cutoff. We have calculated the delta factor for each element for Ecut=10 Ha, 12 Ha, 15 Ha, 17.5 Ha, 20 Ha, 25 Ha and 40 Ha. The following table shows the absolute difference of the $\Delta_1$ factor in meV compared to the converged value at Ecut=40 Ha.
\begin{verbatim}

# Ecut	10 Ha    12 Ha	  15 Ha	  17.5 Ha	  20 Ha	   25 Ha	   40 Ha 

 H	   8.496	   8.131	   5.986	   3.257	   1.304	  -0.088	   0.000
He	  26.637	  75.749	   2.172	   3.652	   1.206	   3.253	   0.000
Li	   0.785	   0.008	   0.650	   0.058	  -0.103	  -0.094	   0.000
Be	   0.591	   0.044	  -0.231	  -0.219	  -0.090	  -0.015	   0.000
 B	   0.072	   1.029	   0.986	   0.367	   0.099	   0.040	   0.000
 C	  18.620	   1.786	  -0.662	  -0.759	  -0.772	  -0.194	   0.000
 N	   4.141	  14.617	  13.676	   3.931	   0.567	  -0.204	   0.000
 O	  70.776	  31.858	   9.273	   0.866	   0.346	   0.282	   0.000
 F	 111.608	  13.347	   7.535	  -0.462	   0.149	  -0.224	   0.000
Ne	  15.490	   7.554	   7.878	  21.914	  -0.957	   0.459	   0.000
Na	     NC		   11.555	  -4.554	   0.189	   0.025	   0.305	   0.000
Mg	   0.418	  -0.135	  -0.102	   0.591	  -0.490	   0.399	   0.000
Al	   3.095	   1.192	  -0.168	  -0.210	  -0.171	  -0.026	   0.000
Si	  -0.453	  -0.344	  -0.090	  -0.092	  -0.026	  -0.017	   0.000
 P	   4.953	   2.296	  -1.333	  -1.013	  -0.591	  -0.044	   0.000
 S	   5.416	   4.192	   0.911	  -0.659	  -0.573	  -0.047	   0.000
Cl	   9.561	   8.118	   2.845	   0.196	  -0.813	  -0.112	   0.000
Ar	  21.102	    NC	 	   0.716	   6.629	   0.244	   0.539	   0.000
 K	   4.136	   4.020	   0.680	  -0.046	  -0.012	  -0.224	   0.000
Ca	   3.294	   0.415	   0.672	   0.448	   0.146	  -0.002	   0.000
Sc	   0.788	   0.495	   0.101	   0.001	  -0.034	  -0.021	   0.000
Ti	  -0.620	   0.333	   0.003	  -0.007	   0.014	  -0.021	   0.000
 V	   0.335	  -0.034	   0.008	  -0.035	  -0.003	  -0.027	   0.000
Cr	   4.227	   8.864	   5.156	   0.205	  -3.626	  -0.780	   0.000
Mn	  49.221	   1.189	   0.293	   1.564	  -0.026	   0.306	   0.000
Fe	  19.818	  -0.151	   0.010	  -0.193	  -0.056	   0.209	   0.000
Co	  30.274	   1.309	   0.120	   1.746	   0.921	   0.309	   0.000
Ni	    NC  	     NC		   19.282	  21.230	  -1.858	  -0.565	   0.000
Cu	  12.082	   7.507	   0.834	  -0.728	  -1.388	  -0.518	   0.000
Zn	  48.953	   0.784	   0.425	   0.004	   0.209	   0.117	   0.000
Ga	   2.906	   0.292	  -0.103	  -0.110	  -0.051	   0.002	   0.000
Ge	  22.142	   4.919	   1.008	   0.328	   0.038	  -0.021	   0.000
As	  -0.595	  -0.161	   0.143	   0.056	  -0.012	  -0.049	   0.000
Se	   0.103	   1.198	   0.129	   0.377	   0.043	   0.007	   0.000
Br	  -0.335	   0.499	  -0.234	  -0.293	  -0.046	   0.011	   0.000
Kr	   0.783	   2.318	   0.215	   0.565	   0.242	  -0.127	   0.000
Rb	   1.886	   0.008	  -0.711	  -0.157	  -0.178	   0.067	   0.000
Sr	   0.761	   0.443	   0.298	  -0.203	   0.022	   0.012	   0.000
 Y	   0.372	   1.031	   0.074	   0.049	  -0.079	  -0.020	   0.000
Zr	   3.456	   0.270	   0.035	   0.054	   0.164	   0.038	   0.000
Nb	   1.211	   0.882	  -0.125	  -0.158	  -0.220	   0.019	   0.000
Mo	   1.091	   0.129	   0.109	  -0.070	  -0.026	  -0.082	   0.000
Tc	   1.047	  -0.534	  -0.229	  -0.116	  -0.062	  -0.011	   0.000
Ru	   0.298	   0.454	  -0.056	  -0.140	  -0.070	   0.061	   0.000
Rh	   1.606	   1.432	  -0.814	   0.293	  -0.296	  -0.337	   0.000
Pd	  21.532	  14.000	   5.090	   0.018	   0.351	   0.295	   0.000
Ag	   2.043	  -0.099	  -0.109	  -0.124	  -0.100	   0.001	   0.000
Cd	   3.727	   0.114	  -0.351	  -0.405	  -0.175	  -0.139	   0.000
In	   8.225	  -0.735	   0.273	  -0.073	  -0.252	  -0.260	   0.000
Sn	   3.254	   1.046	   1.291	   0.439	  -0.329	   0.429	   0.000
Sb	  -0.327	  -0.218	   0.022	   0.031	   0.039	  -0.010	   0.000
Te	   0.505	   0.151	  -0.108	  -0.033	  -0.021	  -0.010	   0.000
 I	   0.296	   0.046	   0.075	   0.066	  -0.011	   0.007	   0.000
Xe	   3.969	   8.409	   7.455	   4.665	  -1.216	   1.554	   0.000
Cs	   5.666	   4.324	   7.542	  -0.026	   0.861	   0.182	   0.000
Ba	   1.170	  -0.097	  -1.551	  -0.206	  -0.335	   0.133	   0.000
Lu	 165.692	   5.865	   0.538	   0.424	   0.091	   0.049	   0.000
Hf	   1.692	   0.080	   0.053	   0.011	   0.062	   0.019	   0.000
Ta	   0.416	   1.417	   0.015	   0.375	   0.250	   0.034	   0.000
 W	   0.712	   0.233	   0.077	   0.097	   0.222	   0.005	   0.000
Re	   1.769	   0.692	   0.108	   0.078	   0.246	   0.015	   0.000
Os	  -0.028	   0.054	   0.019	   0.010	   0.024	   0.004	   0.000
Ir	   1.234	  -0.155	  -0.165	  -0.165	  -0.068	   0.041	   0.000
Pt	  18.814	   5.866	   0.787	   0.724	   0.343	   0.101	   0.000
Au	  29.222	   7.812	   1.280	   0.992	   0.318	   0.122	   0.000
Hg	 120.031	  21.774	   0.152	   0.009	  -0.071	  -0.115	   0.000
Tl	   2.111	   0.488	   0.205	   0.605	  -0.085	  -0.094	   0.000
Pb	   1.363	  -0.325	  -0.254	  -0.281	  -0.117	  -0.014	   0.000
Bi	   0.808	   0.297	   0.242	   0.156	   0.032	   0.019	   0.000
Po	   0.035	   0.056	   0.164	  -0.054	   0.004	   0.028	   0.000
Rn	   2.910	   3.614	   0.041	   0.374	  -0.031	   0.928	   0.000



\end{verbatim}

From this table we have calculated recommended values for the plane wave cutoff energy:\\
Low value:  abs($\Delta_1$-$\Delta_{1}(40Ha)$) $<$ 5 meV\\
Medium value: abs($\Delta_1$-$\Delta_{1}(40Ha)$) $<$ 2 meV\\
High value: abs($\Delta_1$-$\Delta_{1}(40Ha)$) $<$ 1 meV\\

These values are given here and are inserted inside each PAW data XML file for each element.
\begin{verbatim}
# Ecut low medium high 

 H  17.50  20.00  25.00
He  15.00  25.00  25.00
Li  10.00  10.00  10.00
Be  10.00  10.00  10.00
 B  10.00  10.00  15.00
 C  12.00  12.00  15.00
 N  17.50  20.00  20.00
 O  17.50  17.50  17.50
 F  17.50  17.50  17.50
Ne  20.00  20.00  20.00
Na  15.00  17.50  17.50
Mg  10.00  10.00  10.00
Al  10.00  12.00  15.00
Si  10.00  10.00  10.00
 P  10.00  15.00  20.00
 S  12.00  15.00  15.00
Cl  15.00  17.50  17.50
Ar  20.00  20.00  20.00
 K  10.00  15.00  15.00
Ca  10.00  12.00  12.00
Sc  10.00  10.00  10.00
Ti  10.00  10.00  10.00
 V  10.00  10.00  10.00
Cr  17.50  25.00  25.00
Mn  12.00  12.00  20.00
Fe  12.00  12.00  12.00
Co  12.00  12.00  20.00
Ni  20.00  20.00  25.00
Cu  15.00  15.00  25.00
Zn  12.00  12.00  12.00
Ga  10.00  12.00  12.00
Ge  12.00  15.00  17.50
As  10.00  10.00  10.00
Se  10.00  10.00  15.00
Br  10.00  10.00  10.00
Kr  10.00  15.00  15.00
Rb  10.00  10.00  12.00
Sr  10.00  10.00  10.00
 Y  10.00  10.00  15.00
Zr  10.00  12.00  12.00
Nb  10.00  10.00  12.00
Mo  10.00  10.00  12.00
Tc  10.00  10.00  12.00
Ru  10.00  10.00  10.00
Rh  10.00  10.00  15.00
Pd  17.50  17.50  17.50
Ag  10.00  12.00  12.00
Cd  10.00  12.00  12.00
In  12.00  12.00  12.00
Sn  10.00  12.00  17.50
Sb  10.00  10.00  10.00
Te  10.00  10.00  10.00
 I  10.00  10.00  10.00
Xe  17.50  20.00  25.00
Cs  17.50  17.50  17.50
Ba  10.00  10.00  17.50
Lu  15.00  15.00  15.00
Hf  10.00  10.00  12.00
Ta  10.00  10.00  15.00
 W  10.00  10.00  10.00
Re  10.00  10.00  12.00
Os  10.00  10.00  10.00
Ir  10.00  10.00  12.00
Pt  15.00  15.00  15.00
Au  15.00  15.00  17.50
Hg  15.00  15.00  15.00
Tl  10.00  12.00  12.00
Pb  10.00  10.00  12.00
Bi  10.00  10.00  10.00
Po  10.00  10.00  10.00
Rn  10.00  15.00  15.00
#n_low(10)= 44 n_low(12)=  8 n_low(15)=  8 n_low(17.5)=  8 n_low(20)=  3 n_low(25)=  0
#n_med(10)= 32 n_med(12)= 14 n_med(15)= 11 n_med(17.5)=  6 n_med(20)=  6 n_med(25)=  2
#n_high(10)= 18 n_high(12)= 17 n_high(15)= 14 n_high(17.5)= 10 n_high(20)=  6 n_high(25)=  6


\end{verbatim}

\section{PAW atomic data validation against fcc, bcc,  rocksalt, perovskite, half-heusler and zinc-blende structures}

Following \cite{web5}, we have calculated the lattice parameters for fcc, bcc, rocksalt, perovskite, half-heusler and zinc-blende structures and compare the results to a WIEN2k calculation in the same conditions. The calculations were performed with the ABINIT code with the following input files:\\
Common part of general input file for all the structures: \\
\begin{verbatim}
ndtset 7
nsym 0
occopt 3
pawovlp -1
prteig 0
prtden 0
prtcif 1
tsmear 0.001
ecutsm 0.5
ecut 20
pawecutdg 40
chkprim 0
usexcnhat -1
ngkpt 8 8 8
chksymbreak 0
paral_kgb 0
nstep 99
toldfe 1.0d-8

getwfk2 -1
getwfk3 -1
getwfk4 -1
getwfk5 -1
getwfk6 -1
getwfk7 -1

scalecart3 3*0.9932883883792687
scalecart2 3*0.986484829732188
scalecart1 3*0.9795861087155615
scalecart5 3*1.006622709560113
scalecart6 3*1.0131594038201772
scalecart7 3*1.0196128224222163
\end{verbatim}

For fcc structures (example for Al):\\
\begin{verbatim}
shiftk 0.0 0.0 0.0
acell  4.040210000000    4.040210000000    4.040210000000  angstrom
xred 0 0 0
rprim 0.0 0.5 0.5
      0.5 0.0 0.5
      0.5 0.5 0.0
natom 1 typat 1
ntypat 1
znucl  13
nband 8
\end{verbatim}

For bcc structures(example for Al):\\
\begin{verbatim}
shiftk 0.5 0.5 0.5
acell  3.240000000000    3.240000000000    3.240000000000  angstrom
xred 0 0 0
rprim -0.5 0.5 0.5
      0.5 -0.5 0.5
      0.5 0.5 -0.5
natom 1 typat 1
ntypat 1
znucl  13
nband 8
\end{verbatim}

For rocksalt structures(example for NaCl):\\
\begin{verbatim}
shiftk 0.0 0.0 0.0
acell  5.714000000000    5.714000000000    5.714000000000  angstrom
xred 0 0 0
 0.5 0.5 0.5
rprim 0.0 0.5 0.5
      0.5 0.0 0.5
      0.5 0.5 0.0
natom 2 typat 1 2
ntypat 2
znucl  11   17
nband 21
\end{verbatim}

For perovskite structures(example for BaTiO3):\\
\begin{verbatim}
shiftk 0.0 0.0 0.0
acell  4.024000000000    4.024000000000    4.024000000000  angstrom
xred 0.5 0.5 0.5
 0 0 0
 0.5 0 0
 0 0.5 0
 0 0 0.5
rprim 1.0 0.0 0.0
      0.0 1.0 0.0
      0.0 0.0 1.0
natom 5 typat 1 2 3 3 3
ntypat 3
znucl  56   22    8
nband 55
\end{verbatim}

For half-heusler structures(example for AgAlGe):\\
\begin{verbatim}
shiftk 0.0 0.0 0.0
acell  6.224000000000    6.224000000000    6.224000000000  angstrom
xred 0.25 0.25 0.25
 0.5 0.5 0.5
 0 0 0
rprim 0.0 0.5 0.5
      0.5 0.0 0.5
      0.5 0.5 0.0
natom 3 typat 3 2 1
ntypat 3
znucl  47   13   32
nband 33
\end{verbatim}

For zinc-blende structures(example for ZnS):\\
\begin{verbatim}
shiftk 0.0 0.0 0.0
acell  5.442518477764    5.442518477764    5.442518477764  angstrom
xred 0 0 0
 0.25 0.25 0.25
rprim 0.0 0.5 0.5
      0.5 0.0 0.5
      0.5 0.5 0.0
natom 2 typat 1 2
ntypat 2
znucl  30   16
nband 23
\end{verbatim}



A summary of the results is presented \label{tab1}. The values for GBRV in ABINIT and the values used for AE reference calculations come from \cite{web5}.  


\begin{table}[H]
\begin{center}
\begin{tabular}{llllllllccc}
\\\hline
 Test& GBRV-Abinit & JTHv1.1-Abinit \\
\hline
 fcc latt. const. (\%)   &  0.13  &  0.13  \\
 bcc  latt. const. (\%)  & 0.15   & 0.14  \\
 rocksalt  latt. const. (\%)  & 0.13   & 0.16  \\
 perovskite  latt. const. (\%)  & 0.09   & 0.14  \\
 half-heusler  latt. const. (\%)  & 0.13   & 0.15  \\
 zinc-blend $\Delta$ (meV/atom)  & 1.2   & 0.94  \\
 zinc-blend $\Delta$1 (meV/atom)  & 2.1   & 1.71  \\

\\\hline
\end{tabular}
\caption{Summary of PAW data files testing (RMS errors reltative to AE calculations and $\Delta$ factor for zinc-blende structures)  }
\label{tab1}
\end{center}
\end{table}


The detailed results for each element is given on Fig. \ref{fig1} for fcc structures, Fig. \ref{fig2} for bcc structures,  Fig. \ref{fig3} for rocksalt structures, Fig. \ref{fig4} for perovskite structures, Fig. \ref{fig5} for half-heusler structures and Fig. \ref{fig6} for zinc-blende structures. 
\\
\\
\\
\\
\\

\begin{figure}[H]
\centering
{\resizebox{13.0cm}{!}
{\includegraphics{fig-fcc.eps}}} \\
\caption{Percent difference in AE versus PAW data calculations for fcc lattice constant}
\label{fig1}
\end{figure}



\begin{figure}[H]
\begin{center}
{\resizebox{13.0cm}{!}
{\rotatebox{0}{\includegraphics{fig-bcc.eps}}}} \\
\caption{Percent difference in AE versus PAW data calculations for bcc lattice constant}
\label{fig2}
\end{center}
\end{figure}

\begin{figure}[H]
\begin{center}
{\resizebox{13.0cm}{!}
{\rotatebox{0}{\includegraphics{fig-rs.eps}}}} \\
\caption{Percent difference in AE versus PAW data calculations for rocksalt lattice constant}
\label{fig3}
\end{center}
\end{figure}

\begin{figure}[H]
\begin{center}
{\resizebox{12.0cm}{!}
{\rotatebox{0}{\includegraphics{fig-per.eps}}}} \\
\caption{Percent difference in AE versus PAW data calculations for perovskite lattice constant}
\label{fig4}
\end{center}
\end{figure}

\begin{figure}[H]
\begin{center}
{\resizebox{6.0cm}{!}
%{\rotatebox{90}{\includegraphics{fig-hh.eps}}}} \\
{\includegraphics[angle=-90]{fig-hh.eps}}}\\
\caption{Percent difference in AE versus PAW data calculations for half-heusler lattice constant}
\label{fig5}
\end{center}
\end{figure}


\begin{figure}[H]
\begin{center}
{\resizebox{13.0cm}{!}
{\rotatebox{0}{\includegraphics{fig-zb.eps}}}} \\
\caption{$\Delta$ factor for zinc- blende structure}
\label{fig6}
\end{center}
\end{figure}

Following \cite{web5}, we have also calculated magnetic moments of transition metal oxides with non-zero magnetic moments at the AE non-spin polarized lattice constant. The magnetic moments are given in $\mu_B$ per primitive cell. The AE results come from \cite{web5}. The calculations have been done with a 12x12x12 k-point mesh \label{tab2}.\\

\begin{table}[h]
\begin{tabular}{llllllllllllcccc}
\\\hline
 Compound& $\mu_{AE}$   &  $\mu_{GBRV-Abinit}$   &   $\mu_{JTHv1.1-Abinit}$ \\
\hline
 VO &  1.32  &  1.27 & 1.24  \\
 CrO & 2.99  &  3.04 & 3.05  \\
 MnO &  3.85  &  3.84 & 3.86  \\
 FeO &  3.83  &  3.84 & 3.86  \\
 CoO &  2.42  &  2.53 & 2.56  \\
 NiO &  1.68  &  1.47 & 1.36  \\
 MoO &  0.54  &  0.53 & 0.49  \\
 TcO &  1.92  &  1.90 & 1.95  \\
 RuO &  1.64  &  1.63 & 1.67  \\
 OsO &  1.56  &  1.50 & 1.68  \\
 IrO &  0.62  &  0.62 & 0.74  \\
\\\hline
\end{tabular}
\caption{Magnetic moments of transition metal oxides  }
\label{tab2}
\end{table}


In addition, we have generated atomic data for rare-earth elements. The lattice parameters found for the fcc struture are the following \label{tab3}:
\begin{table}[h]
\begin{tabular}{llcccccccccccccccc}
\\\hline
Element & La & Ce & Pr & Nd & Pm & Sm & Eu & Gd & Tb & Dy & Ho & Er & Tm & Yb & Lu  \\
fcc lattice parameter (\AA) & 5.272 & 4.769 & 4.608 & 4.531 & 4.499 & 4.516 & 4.630 & 4.723 & 4.835 & 4.918 & 5.012 & 5.073 & 5.120 & 5.163 & 4.867  \\
\\\hline
\end{tabular}
\caption{fcc lattice parameters obtained with JTHv1.1 atomic data in Abinit for the rare-earth elements }
\label{tab3}
\end{table}


\section{Conclusions}

 The JTHv1.1 table has good accuracy and efficiency compared to other similar datasets. It makes it a good candidate for high-throughput calculations. This new table is provided as XML files, that makes it easily readable by all PAW codes (ABINIT, GPAW, PWPAW, SOCORRO, ...). It is distributed on the ABINIT web site \cite{web3}. 

\section{Acknowledgments}

This work was performed using HPC resources from the French Research and Technology Computing Center (CCRT).  



%\bibliographystyle{model1-num-names}
\bibliography{JTH}




\end{document}


